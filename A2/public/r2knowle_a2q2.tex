% <- percent signs are used to comment
\documentclass[12pt]{article}

%%%%%% PACKAGES - this part loads additional material for LaTeX %%%%%%%%%
% Nearly anything you want can be done in LaTeX if you load the right package 
% (search ctan.org or google it if you are looking for something).  We will load
% here a few that we need for this document or that we expect you to need later.

% The next 3 lines are needed to fix shortcomings of TeX that only make sense given its 40-year history ...
% Simple keep and ignore.
\usepackage[utf8]{inputenc}
\usepackage[T1]{fontenc}
\usepackage{lmodern}
\usepackage{amsmath}
\usepackage{changepage}
\usepackage{lipsum}
\usepackage{caption}
\usepackage{qtree}
\usepackage{algorithm}
\usepackage{algpseudocode}
\usepackage{algorithmicx}
\usepackage[section]{placeins}
% Custom margins (and paper sizes etc.) because LaTeX else wastes much space
\usepackage[margin=1in]{geometry}

% The following packages are created by the American Mathematical Society (AMS)
% and provide lots of tools for special fonts, symbols, theorems, and proof
\usepackage{amsmath,amsfonts,amssymb,amsthm}
% mathtools contains many detail improvements over ams and core tex
\usepackage{mathtools}

% graphicx is required for images
\usepackage{graphicx}

% enumitem used for customizing enumerations
\usepackage[shortlabels]{enumitem}

% tikz is the package used for drawing, in particular for drawing trees. You may also find simplified packages like tikz-qtree and forest useful
\usepackage{tikz}

% hyperref allows links, urls, and many other PDF tricks.  We load it here
%          in such a way that the PDF file has info about it
\usepackage[%
	pdftitle={CS240 Assignment 0},%
	hidelinks,%
]{hyperref}


%%%%%% COMMANDS - here you can define your own LaTeX-commands %%%%%%%%%

%%%%%% End of Preamble %%%%%%%%%%%%%

\begin{document}

\begin{center}
{\Large\textbf{CS370, Winter 2023}}\\
\vspace{2mm}
{\Large\textbf{Assignment 2: Question 2}}\\
\vspace{3mm}
\end{center}

\begin{adjustwidth}{0em}{0pt}

\section*{(a) Finding the conditionals for a,b,c,e,f,g}
We are told that our piecewise function must pass through (1,2), (2,1), (3,1). To begin we will solve for a,e b by plugging points into certain functions. Lets use (1,2) and plug it into S(x):
\begin{align*}
   \text{S(1)} &= a + b(x-1) + c(x-1)^2 + \frac{1}{4}(x-1)^2(x-2) \\
             2 &= a + b((1)-1) + c((1)-1)^2 + \frac{1}{4}((1)-1)^2((1)-2)) \\
             2 &= a \\
             a &= 2 
\end{align*}
Plugging (2,1) into a equation for S(x) we then get:
\begin{align*}
   \text{S(2)} &= e + f(x-2) + g(x-2)^2 - \frac{1}{4}(x-2)^2(x-3) \\
             1 &= e + f((2)-2) + g((2)-2)^2 - \frac{1}{4}((2)-2)^2((2)-3) \\
             1 &= e \\
             e &= 1 
\end{align*}
If we plug (2,1) into the other equation for S(x) it tells us the conditions for the other variables:
\begin{align*}
   \text{S(2)} &= a + b(x-1) + c(x-1)^2 + \frac{1}{4}(x-1)^2(x-2) \\
             1 &= a + b((2)-1) + c((2)-1)^2 + \frac{1}{4}((2)-1)^2((2)-2)) \\
             1 &= a + b + c \\
             b + c &= -1
\end{align*}
If we plug (3,1) into the equation for S(x) we get:
\begin{align*}
   \text{S(3)} &= e + f(x-2) + g(x-2)^2 - \frac{1}{4}(x-2)^2(x-3) \\
             1 &= e + f((3)-2) + g((3)-2)^2 - \frac{1}{4}((3)-2)^2((3)-3) \\
             1 &= e + f + g \\
             f + g &= 0
\end{align*}
Which gives us the conditions for b,c,f,g and the exact values for a and e.
\section*{(b) Find condition of coefficients such that S'(2) is 0}
In order for equation to be continuous both equations for S'(2) must be equal. This implies that we have the following:
\begin{align*}
   \text{S'(2)} &= \text{S'(2)} \\
             \frac{dS}{dx} (a + b(x-1) + c(x-1)^2 + \frac{1}{4}(x-1)^2(x-2)) &= \frac{dS}{dx} (e + f(x-2) + g(x-2)^2 - \frac{1}{4}(x-2)^2(x-3)) \\
             b + 2c(x-1) + \frac{(x-1)(3x-5)}{4} &= f + 2g(x-2) - \frac{(x-2)(3x-8)}{4}) \\
             b + 2c((2)-1) + \frac{((2)-1)(3(2)-5)}{4} &= f + 2g((2)-2) - \frac{((2)-2)(3(2)-8)}{4}) \\
             b + 2c + \frac{1}{4} &= f \\
             b + 2c - f &= -\frac{1}{4} \\
\end{align*}
\section*{(c) Enforce the boundary conditions and solve for c,g}
To begin with we are told that the second derivative of S(1) and S(3) is 0. To get an equation for c we will then rewrite this as:
\begin{align*}
   \text{S(1)} &= \frac{d^2S}{d^2x} (a + b(x-1) + c(x-1)^2 + \frac{1}{4}(x-1)^2(x-2)) \\
             0 &= 2c + \frac{3x - 4}{2} \\
             0 &= 2c + \frac{3(1) - 4}{2} \\
             0 &= 2c - \frac{1}{2} \\
             c &= \frac{1}{4} \\\\
    \text{S(3)} &= \frac{d^2S}{d^2x} (e + f(x-2) + g(x-2)^2 - \frac{1}{4}(x-2)^2(x-3)) \\
             0 &= g - \frac{3x - 7}{2} \\
             0 &= g - \frac{3(3) - 7}{2} \\
             0 &= g - 1 \\
             g &= \frac{1}{2}
\end{align*}
Which proves the coefficients values as necessary. 



\section*{(d) Compute the values of b and f}
Now that we have the values for g and c from (c), we can plug it into the following equations:
\begin{align*}
   b + c &= -1 \\
             b + \frac{1}{4} &= -1 \\
             b  &= -\frac{5}{4} \\\\
   f + g &= 0 \\
             f + \frac{1}{2} &= 0 \\
             f  &= -\frac{1}{2} \\
\end{align*}
Thus we have have solved for both f and b.

\section*{(e) What conditions are needed for a cubic spline}
We need to check that S''(x) is continuous when x = 2.
\end{adjustwidth}

\end{document}
