% <- percent signs are used to comment
\documentclass[12pt]{article}

%%%%%% PACKAGES - this part loads additional material for LaTeX %%%%%%%%%
% Nearly anything you want can be done in LaTeX if you load the right package 
% (search ctan.org or google it if you are looking for something).  We will load
% here a few that we need for this document or that we expect you to need later.

% The next 3 lines are needed to fix shortcomings of TeX that only make sense given its 40-year history ...
% Simple keep and ignore.
\usepackage[utf8]{inputenc}
\usepackage[T1]{fontenc}
\usepackage{lmodern}
\usepackage{amsmath}
\usepackage{changepage}
\usepackage{lipsum}
\usepackage{caption}
\usepackage{qtree}
\usepackage{algorithm}
\usepackage{algpseudocode}
\usepackage{algorithmicx}
\usepackage[section]{placeins}
\usepackage{forest}
% Custom margins (and paper sizes etc.) because LaTeX else wastes much space
\usepackage[margin=1in]{geometry}

% The following packages are created by the American Mathematical Society (AMS)
% and provide lots of tools for special fonts, symbols, theorems, and proof
\usepackage{amsmath,amsfonts,amssymb,amsthm}
% mathtools contains many detail improvements over ams and core tex
\usepackage{mathtools}

% graphicx is required for images
\usepackage{graphicx}
\usetikzlibrary{positioning,chains,fit,shapes,calc, arrows}
\newenvironment{amatrix}[1]{%
  \left(\begin{array}{@{}*{#1}{c}|c@{}}
}{%
  \end{array}\right)
}
% enumitem used for customizing enumerations
\usepackage[shortlabels]{enumitem}

% tikz is the package used for drawing, in particular for drawing trees. You may also find simplified packages like tikz-qtree and forest useful
\usepackage{tikz}
\definecolor{myblue}{RGB}{80,80,160}
\definecolor{mygreen}{RGB}{80,160,80}

% hyperref allows links, urls, and many other PDF tricks.  We load it here
%          in such a way that the PDF file has info about it
\usepackage[%
	pdftitle={CS370 Assignment 0},%
	hidelinks,%
]{hyperref}


%%%%%% COMMANDS - here you can define your own LaTeX-commands %%%%%%%%%

%%%%%% End of Preamble %%%%%%%%%%%%%

\begin{document}

\begin{center}
{\Large\textbf{CS370, Winter 2023}}\\
\vspace{2mm}
{\Large\textbf{Assignment 5: Question 6}}\\
\vspace{3mm}
\end{center}

\begin{adjustwidth}{0em}{0pt}
\textbf{Q5a)} Provide a solution to the system of equations \\\\
A accurate solution to the system of equations is:
\[ 
\begin{bmatrix}
4 & -4.2 & 0\\
-2 & 2.099 & 7 \\
4 & -1.7 & 4 \\
\end{bmatrix}
\begin{bmatrix}
0 \\
-1 \\
1 \\
\end{bmatrix}
=
\begin{bmatrix}
4.2 \\
4.901 \\
5.7 \\
\end{bmatrix}\]
and so our solution is:
\[ x_0 = 0, x_1 = -1, x_2 = 1\]
\textbf{Q5b)} Provide a solution to the system of equations using Gaussian elimination with no row pivoting \\\\
To begin we will start with the matrix:
\[ 
\begin{bmatrix}
4 & -4.2 & 0\\
-2 & 2.099 & 7 \\
4 & -1.7 & 4 \\
\end{bmatrix}, \begin{bmatrix}
4.2 \\
4.901 \\
5.7 \\
\end{bmatrix}\]
We will then add 1/2(1) into (2):
\[ 
\begin{bmatrix}
4 & -4.2 & 0\\
0 & -0.001 & 7 \\
4 & -1.7 & 4 \\
\end{bmatrix}, \begin{bmatrix}
4.2 \\
7.001 \\
5.7 \\
\end{bmatrix}\]
We will then subtract (1) into (3):
\[ 
\begin{bmatrix}
4 & -4.2 & 0\\
0 & -0.001 & 7 \\
0 & 2.5 & 4 \\
\end{bmatrix} =,
\begin{bmatrix}
4.2 \\
7.001 \\
1.5 \\
\end{bmatrix}\]
We will then add 2500(2) to (3):
\[ 
\begin{bmatrix}
4 & -4.2 & 0\\
0 & -0.001 & 7 \\
0 & 0 & 17504 \\
\end{bmatrix}, \begin{bmatrix}
4.2 \\
7.001 \\
17503 (rounding down) \\
\end{bmatrix}\]
We can now use back substitution starting with solving $x_3$ which gives us:
\[ 170504x_2 = 170503 \]
\[ x_2 = 0.9999 \]
Using this we thus get:
\[ -0.001x_1 + 7(0.9999) = 7.001 \]
\[ x_1 = -1.7\]
We can finally get:
\[ 4x_0 - 4.2(-1.7) = 4.2 \]
\[ x_0 = -0.735\]
Which is very far off from our original estimate as we have:
\[ x_0 = -0.735, -1.7, 0.9999 \]

\textbf{Q5b)} Provide a solution to the system of equations using Gaussian elimination with row pivoting \\\\
To begin we will start with the matrix:
\[ 
\begin{bmatrix}
4 & -4.2 & 0\\
-2 & 2.099 & 7 \\
4 & -1.7 & 4 \\
\end{bmatrix}, \begin{bmatrix}
4.2 \\
4.901 \\
5.7 \\
\end{bmatrix}\]
We will then switch (1) and (3):
\[ 
\begin{bmatrix}
4 & -1.7 & 4 \\
-2 & 2.099 & 7 \\
4 & -4.2 & 0\\
\end{bmatrix}, \begin{bmatrix}
5.7 \\
4.901 \\
4.2 \\
\end{bmatrix}\]
We will add 1/2(1) to (2):
\[ 
\begin{bmatrix}
4 & -1.7 & 4 \\
0 & 1.249 & 9 \\
4 & -4.2 & 0\\
\end{bmatrix}, \begin{bmatrix}
5.7 \\
7.751 \\
4.2 \\
\end{bmatrix}\]
We will then subtract (1) to (3):
\[ 
\begin{bmatrix}
4 & -1.7 & 4 \\
0 & 1.249 & 9 \\
0 & -2.5 & -4\\
\end{bmatrix}, \begin{bmatrix}
5.7 \\
7.751 \\
-1.5 \\
\end{bmatrix}\]
We will then multiply by 2.0016 (2) to (3):
\[ 
\begin{bmatrix}
4 & -1.7 & 4 \\
0 & 1.249 & 9 \\
0 & 0 & 14.014\\
\end{bmatrix}, \begin{bmatrix}
5.7 \\
7.751 \\
14.014 \\
\end{bmatrix}\]
Solving now we thus get:
\[14.014x_2 = 14.014\]
\[x_2 = 1\]
Plugging this in to the next equation we get:
\[ 1.249x_1 + 9 = 7.751\]
\[ x_1 = -1 \]
Lastly we get:
\[ 4x_0 + 4.2 = 4.2 \]
\[ x_0 = 0 \]
Which is very similar to the accurate answer we had at the start.
\textbf{Q5a)} Which system was more accurate \\\\
In this case it would appear that system c is more accurate then system b, and by switching the rows we were able to avoid the rounding error!
\end{adjustwidth}



\end{document}