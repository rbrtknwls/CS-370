% <- percent signs are used to comment
\documentclass[12pt]{article}

%%%%%% PACKAGES - this part loads additional material for LaTeX %%%%%%%%%
% Nearly anything you want can be done in LaTeX if you load the right package 
% (search ctan.org or google it if you are looking for something).  We will load
% here a few that we need for this document or that we expect you to need later.

% The next 3 lines are needed to fix shortcomings of TeX that only make sense given its 40-year history ...
% Simple keep and ignore.
\usepackage[utf8]{inputenc}
\usepackage[T1]{fontenc}
\usepackage{lmodern}
\usepackage{amsmath}
\usepackage{changepage}
\usepackage{lipsum}
\usepackage{caption}
\usepackage{qtree}
\usepackage{algorithm}
\usepackage{algpseudocode}
\usepackage{algorithmicx}
\usepackage[section]{placeins}
\usepackage{forest}
% Custom margins (and paper sizes etc.) because LaTeX else wastes much space
\usepackage[margin=1in]{geometry}

% The following packages are created by the American Mathematical Society (AMS)
% and provide lots of tools for special fonts, symbols, theorems, and proof
\usepackage{amsmath,amsfonts,amssymb,amsthm}
% mathtools contains many detail improvements over ams and core tex
\usepackage{mathtools}

% graphicx is required for images
\usepackage{graphicx}
\usetikzlibrary{positioning,chains,fit,shapes,calc, arrows}

% enumitem used for customizing enumerations
\usepackage[shortlabels]{enumitem}

% tikz is the package used for drawing, in particular for drawing trees. You may also find simplified packages like tikz-qtree and forest useful
\usepackage{tikz}
\definecolor{myblue}{RGB}{80,80,160}
\definecolor{mygreen}{RGB}{80,160,80}

% hyperref allows links, urls, and many other PDF tricks.  We load it here
%          in such a way that the PDF file has info about it
\usepackage[%
	pdftitle={CS370 Assignment 0},%
	hidelinks,%
]{hyperref}


%%%%%% COMMANDS - here you can define your own LaTeX-commands %%%%%%%%%

%%%%%% End of Preamble %%%%%%%%%%%%%

\begin{document}

\begin{center}
{\Large\textbf{CS370, Winter 2023}}\\
\vspace{2mm}
{\Large\textbf{Assignment 5: Question 4}}\\
\vspace{3mm}
\end{center}

\begin{adjustwidth}{0em}{0pt}
\textbf{Q4a)} Write the coefficient system matrix. \\\\
We can convert the system of equations we are given to get:
\[ 
\begin{bmatrix}
-24 & 12 & 36 & -12\\
-12 & 30 & -30 & -18\\
-12 & -2 & 40 & 22\\
6 & -15 & 3 & 33
\end{bmatrix}
\begin{bmatrix}
x_0 \\
x_1 \\
x_2 \\
x_3 \\
\end{bmatrix}
=
\begin{bmatrix}
36 \\
-18 \\
18 \\
-39
\end{bmatrix}\]
Where A is the left most matrix, and b is the right most vector.\\\\
\textbf{Q4a)} Compute the LU factorization of A: \\\\
To begin with we will combine A with the unit-diagonal matrix as they are equivalent:
\[ 
\begin{bmatrix}
-24 & 12 & 36 & -12\\
-12 & 30 & -30 & -18\\
-12 & -2 & 40 & 22\\
6 & -15 & 3 & 33
\end{bmatrix}
=
\begin{bmatrix}
1 & 0 & 0 & 0\\
0 & 1 & 0 & 0\\
0 & 0 & 1 & 0\\
0 & 0 & 0 & 1
\end{bmatrix}
\begin{bmatrix}
-24 & 12 & 36 & -12\\
-12 & 30 & -30 & -18\\
-12 & -2 & 40 & 22\\
6 & -15 & 3 & 33
\end{bmatrix}\]
We will then subtract the second row by half of the first row:
\[
=
\begin{bmatrix}
1 & 0 & 0 & 0\\
\frac{1}{2} & 1 & 0 & 0\\
0 & 0 & 1 & 0\\
0 & 0 & 0 & 1
\end{bmatrix}
\begin{bmatrix}
-24 & 12 & 36 & -12\\
0 & 24 & -48 & -12\\
-12 & -2 & 40 & 22\\
6 & -15 & 3 & 33
\end{bmatrix}\]
We will then subtract the third row by half of the first row:
\[
=
\begin{bmatrix}
1 & 0 & 0 & 0\\
\frac{1}{2} & 1 & 0 & 0\\
\frac{1}{2} & 0 & 1 & 0\\
0 & 0 & 0 & 1
\end{bmatrix}
\begin{bmatrix}
-24 & 12 & 36 & -12\\
0 & 24 & -48 & -12\\
0 & -8 & 22 & 28\\
6 & -15 & 3 & 33
\end{bmatrix}\]
We will then subtract the negative quarter of the first row to the fourth row:
\[
=
\begin{bmatrix}
1 & 0 & 0 & 0\\
\frac{1}{2} & 1 & 0 & 0\\
\frac{1}{2} & 0 & 1 & 0\\
-\frac{1}{4} & 0 & 0 & 1
\end{bmatrix}
\begin{bmatrix}
-24 & 12 & 36 & -12\\
0 & 24 & -48 & -12\\
0 & -8 & 22 & 28\\
0 & -12 & 12 & 30
\end{bmatrix}\]
We will then subtract a negative third of the first second to the third row:
\[
=
\begin{bmatrix}
1 & 0 & 0 & 0\\
\frac{1}{2} & 1 & 0 & 0\\
\frac{1}{2} & -\frac{1}{3} & 1 & 0\\
-\frac{1}{4} & 0 & 0 & 1
\end{bmatrix}
\begin{bmatrix}
-24 & 12 & 36 & -12\\
0 & 24 & -48 & -12\\
0 & 0 & 6 & 24\\
0 & -12 & 12 & 30
\end{bmatrix}\]
We will then subtract a negative half of the second row to the fourth row:
\[
=
\begin{bmatrix}
1 & 0 & 0 & 0\\
\frac{1}{2} & 1 & 0 & 0\\
\frac{1}{2} & -\frac{1}{3} & 1 & 0\\
-\frac{1}{4} & -\frac{1}{2} & 0 & 1
\end{bmatrix}
\begin{bmatrix}
-24 & 12 & 36 & -12\\
0 & 24 & -48 & -12\\
0 & 0 & 6 & 24\\
0 & 0 & -12 & 24
\end{bmatrix}\]
We will then subtract 2 times the third row from the fourth row:
\[
=
\begin{bmatrix}
1 & 0 & 0 & 0\\
\frac{1}{2} & 1 & 0 & 0\\
\frac{1}{2} & -\frac{1}{3} & 1 & 0\\
-\frac{1}{4} & -\frac{1}{2} & -2 & 1
\end{bmatrix}
\begin{bmatrix}
-24 & 12 & 36 & -12\\
0 & 24 & -48 & -12\\
0 & 0 & 6 & 24\\
0 & 0 & 0 & 72
\end{bmatrix}\]
This gives us our L (lower triangular and unit diagonal) matrix on the left and our U (upper triangular) matrix on the right. Since we didnt need to do any swaps, P is still unit diagonal 
\textbf{Q4a)} Solve the system using the LU factorization: \\\\
Plugging LU into A we get:
\[ 
\begin{bmatrix}
1 & 0 & 0 & 0\\
\frac{1}{2} & 1 & 0 & 0\\
\frac{1}{2} & -\frac{1}{3} & 1 & 0\\
-\frac{1}{4} & -\frac{1}{2} & -2 & 1
\end{bmatrix}
\begin{bmatrix}
-24 & 12 & 36 & -12\\
0 & 24 & -48 & -12\\
0 & 0 & 6 & 24\\
0 & 0 & 0 & 72
\end{bmatrix}
\begin{bmatrix}
x_0 \\
x_1 \\
x_2 \\
x_3 \\
\end{bmatrix}
=
\begin{bmatrix}
36 \\
-18 \\
18 \\
-39
\end{bmatrix}\]
Lets now assume that U*X = Y, such that:
\[ 
\begin{bmatrix}
-24 & 12 & 36 & -12\\
0 & 24 & -48 & -12\\
0 & 0 & 6 & 24\\
0 & 0 & 0 & 72
\end{bmatrix}
\begin{bmatrix}
x_0 \\
x_1 \\
x_2 \\
x_3 \\
\end{bmatrix}
=
\begin{bmatrix}
y_1 \\
y_2 \\
y_3 \\
y_4
\end{bmatrix}\]
Plugging this back into our equation we get:
\[ 
\begin{bmatrix}
1 & 0 & 0 & 0\\
\frac{1}{2} & 1 & 0 & 0\\
\frac{1}{2} & -\frac{1}{3} & 1 & 0\\
-\frac{1}{4} & -\frac{1}{2} & -2 & 1
\end{bmatrix}
\begin{bmatrix}
y_0 \\
y_1 \\
y_2 \\
y_3 \\
\end{bmatrix}
=
\begin{bmatrix}
36 \\
-18 \\
18 \\
-39
\end{bmatrix}\]
Solving this, by expanding out the system of equations we then get
\[ y_0 = 36, y_1 = -36, y_2 = -12, y_3 = -72 \]
Plugging this back into our equation for Y we then get:
\[ 
\begin{bmatrix}
-24 & 12 & 36 & -12\\
0 & 24 & -48 & -12\\
0 & 0 & 6 & 24\\
0 & 0 & 0 & 72
\end{bmatrix}
\begin{bmatrix}
x_0 \\
x_1 \\
x_2 \\
x_3 \\
\end{bmatrix}
=
\begin{bmatrix}
36 \\
-36 \\
-12 \\
-72
\end{bmatrix}\]
Solving this directly again gives us the following values for x:
\[ x_0 = 3, x_1 = 2, x_2 = 2, x_3 = -1 \]
As required!
\end{adjustwidth}



\end{document}