% <- percent signs are used to comment
\documentclass[12pt]{article}

%%%%%% PACKAGES - this part loads additional material for LaTeX %%%%%%%%%
% Nearly anything you want can be done in LaTeX if you load the right package 
% (search ctan.org or google it if you are looking for something).  We will load
% here a few that we need for this document or that we expect you to need later.

% The next 3 lines are needed to fix shortcomings of TeX that only make sense given its 40-year history ...
% Simple keep and ignore.
\usepackage[utf8]{inputenc}
\usepackage[T1]{fontenc}
\usepackage{lmodern}
\usepackage{amsmath}
\usepackage{changepage}
\usepackage{lipsum}
\usepackage{caption}
\usepackage{qtree}
\usepackage{algorithm}
\usepackage{algpseudocode}
\usepackage{algorithmicx}
\usepackage[section]{placeins}
\usepackage{forest}
% Custom margins (and paper sizes etc.) because LaTeX else wastes much space
\usepackage[margin=1in]{geometry}

% The following packages are created by the American Mathematical Society (AMS)
% and provide lots of tools for special fonts, symbols, theorems, and proof
\usepackage{amsmath,amsfonts,amssymb,amsthm}
% mathtools contains many detail improvements over ams and core tex
\usepackage{mathtools}

% graphicx is required for images
\usepackage{graphicx}
\usetikzlibrary{positioning,chains,fit,shapes,calc, arrows}

% enumitem used for customizing enumerations
\usepackage[shortlabels]{enumitem}

% tikz is the package used for drawing, in particular for drawing trees. You may also find simplified packages like tikz-qtree and forest useful
\usepackage{tikz}
\definecolor{myblue}{RGB}{80,80,160}
\definecolor{mygreen}{RGB}{80,160,80}

% hyperref allows links, urls, and many other PDF tricks.  We load it here
%          in such a way that the PDF file has info about it
\usepackage[%
	pdftitle={CS370 Assignment 0},%
	hidelinks,%
]{hyperref}


%%%%%% COMMANDS - here you can define your own LaTeX-commands %%%%%%%%%

%%%%%% End of Preamble %%%%%%%%%%%%%

\begin{document}

\begin{center}
{\Large\textbf{CS370, Winter 2023}}\\
\vspace{2mm}
{\Large\textbf{Assignment 5: Question 5}}\\
\vspace{3mm}
\end{center}

\begin{adjustwidth}{0em}{0pt}
\textbf{Q5)} How to solve the system xA= b in $O(N^2)$ flops \\\\
In class we are told that to generate the LU factorization takes $O(N^3)$ flops and that back substitution and forward substitution takes $O(N^2)$ flops. To begin with we are given the LU factorization for A, so our equation can immediately become:
\[ xLU = b \]
We can then define a new vector y and define it such that:
\[ y = xL \]
For now we will leave y populated with variables $y_0, y_1 ... y_n$. We can substitute this back into our equation to get:
\[ yU = b \]
Since y and b are both vector and U is a upper triangular matrix, we can use back substitution to solve for Y in O($N^2$) flops are proved in class. Now that we have solved Y we can return to our previous equation:
\[ y = xL \]
Since we know y now, and x is a vector and L is a lower triangular matrix. We can solve for x using forward substitution which will give us our solution in O($N^2$). Thus our this algorithm will solve for x in O($N^2$) flops

\end{adjustwidth}



\end{document}