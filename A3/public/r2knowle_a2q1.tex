% <- percent signs are used to comment
\documentclass[12pt]{article}

%%%%%% PACKAGES - this part loads additional material for LaTeX %%%%%%%%%
% Nearly anything you want can be done in LaTeX if you load the right package 
% (search ctan.org or google it if you are looking for something).  We will load
% here a few that we need for this document or that we expect you to need later.

% The next 3 lines are needed to fix shortcomings of TeX that only make sense given its 40-year history ...
% Simple keep and ignore.
\usepackage[utf8]{inputenc}
\usepackage[T1]{fontenc}
\usepackage{lmodern}
\usepackage{amsmath}
\usepackage{changepage}
\usepackage{lipsum}
\usepackage{caption}
\usepackage{qtree}
\usepackage{algorithm}
\usepackage{algpseudocode}
\usepackage{algorithmicx}
\usepackage[section]{placeins}
% Custom margins (and paper sizes etc.) because LaTeX else wastes much space
\usepackage[margin=1in]{geometry}

% The following packages are created by the American Mathematical Society (AMS)
% and provide lots of tools for special fonts, symbols, theorems, and proof
\usepackage{amsmath,amsfonts,amssymb,amsthm}
% mathtools contains many detail improvements over ams and core tex
\usepackage{mathtools}

% graphicx is required for images
\usepackage{graphicx}

% enumitem used for customizing enumerations
\usepackage[shortlabels]{enumitem}

% tikz is the package used for drawing, in particular for drawing trees. You may also find simplified packages like tikz-qtree and forest useful
\usepackage{tikz}

% hyperref allows links, urls, and many other PDF tricks.  We load it here
%          in such a way that the PDF file has info about it
\usepackage[%
	pdftitle={CS240 Assignment 0},%
	hidelinks,%
]{hyperref}


%%%%%% COMMANDS - here you can define your own LaTeX-commands %%%%%%%%%

%%%%%% End of Preamble %%%%%%%%%%%%%

\begin{document}

\begin{center}
{\Large\textbf{CS370, Winter 2023}}\\
\vspace{2mm}
{\Large\textbf{Assignment 2: Question 2}}\\
\vspace{3mm}
\end{center}

\begin{adjustwidth}{0em}{0pt}

\section*{(1) Do two Euler steps for the following IVP:}
To be being we will define the following variables,$z_1 = x$, $z_2 = y$. From these we can derive that:
\begin{align*}
   \frac{dz_1(t)}{dt} &= 4z_1 - 2z_2 + t \\\\
   \frac{dz_2(t)}{dt} &= 3z_1 + 5t 
\end{align*}
We can then derive $f$ by doing the following:
\[ f(t,z) = \frac{d}{dt} \begin{bmatrix}
z_1\\
z_2
\end{bmatrix} =
\begin{bmatrix}
4z_1 - 2z_2 + t\\
3z_1 + 5t 
\end{bmatrix} \]
Solving for $t = 1.2$ (first step) gives us:
\begin{align*} 
	Z^{(1)} &= Z^{(0)} + hf(t_0, Z^{(0)}) \\
	        &= \begin{bmatrix}
1\\
1
\end{bmatrix} + 0.2\begin{bmatrix}
4(1) - 2(1) + 1\\
3(1) + 5(1) 
\end{bmatrix} \\
	        &= \begin{bmatrix}
1\\
1
\end{bmatrix} + 0.2\begin{bmatrix}
3\\
8 
\end{bmatrix} \\
	        &= \begin{bmatrix}
1.6\\
2.6
\end{bmatrix}
\end{align*}
Solving now for $t = 1.4$ (second step) gives us:
\begin{align*} 
	Z^{(2)} &= Z^{(1)} + hf(t_0, Z^{(1)}) \\
	        &= \begin{bmatrix}
1.6\\
2.6
\end{bmatrix} + 0.2\begin{bmatrix}
4(1.6) - 2(2.6) + 1.2\\
3(1.6) + 5(1.2) 
\end{bmatrix} \\
	        &= \begin{bmatrix}
1.6\\
2.6
\end{bmatrix} + 0.2\begin{bmatrix}
2.4\\
10.8
\end{bmatrix} \\
	        &= \begin{bmatrix}
2.08\\
4.76
\end{bmatrix}
\end{align*}
\end{adjustwidth}
\end{document}
